\chapter{有理标准形}

在\nameref{sec:实数域上的若当标准形}一节中,我们说明了在实数域这样的非代数闭域上,我们不一定能得到若当标准形. 事实上,我们希望能得到一个在任何数域上都存在的标准形,由于有理数域是最小的数域,因此这种标准形被称为有理标准形. 我们将遵循上一讲\autoref{thm:初等因子组是相似全系不变量} 后提出的三个问题的思路来解决构造有理标准形的问题. 为了阅读方便,我们将这三个问题再次叙述:
\begin{enumerate}
    \item 为什么叫初等因子分解?这是因为这样的分解已经是最细的分解了吗(即不能再进一步分解使其与另一种循环子空间分解同构吗)?
    \item 是否还有其它更粗粒度的分解与其它可能的循环子空间分解对应?即我们是否能结合一些初等因子,得到更大的循环子空间?是否有最粗粒度的分解?
    \item 我们知道在复数域上多项式是可以完全分裂成一次多项式的乘积的,但是在其它一些域,例如实数域或者有理数域上,多项式无法完全分裂,那么将会得到二次多项式或者其它形式的初等因子,此时初等因子分解是否还有意义?
\end{enumerate}

\section{推广的广义特征子空间与第二有理标准形}
为了逻辑上的顺畅,我们首先介绍第二有理标准形的构造,因为这一构造恰好需要对若当标准形的推导中广义特征空间以及初等因子的概念进行推广. 但需要注意的是,有理标准形在复数域上并不能退化为若当标准形,即有理标准形并不是在任何数域上都能找到的一个更广泛的标准形概念,不是若当标准形的推广,这一点是初学时必须要注意的,我们也将在后面详细讲述其中的差异与关联.

回顾\nameref{thm:多项式的唯一分解定理}叙述的唯一分解性质,任意数域上的特征多项式(所以是首一的)$f(\lambda)$都可以被唯一分解为一些首一不可约多项式的乘积,即
\[f(\lambda)=p_1(\lambda)^{n_1}p_2(\lambda)^{n_2}\cdots p_k(\lambda)^{n_k},\]
其中$p_i(\lambda)$是首一不可约多项式,$n_i$是正整数. 如果我们讨论的是复数域,则这些不可约多项式都是一次多项式,即$p_i(\lambda)=\lambda-\lambda_i$,其中$\lambda_i$是复数域上的特征值. 但是如果我们讨论的是实数域或者有理数域,则这些不可约多项式可能是一次多项式,也可能是更高次的多项式. 事实上,在复数域上我们推导若当标准形的起点是定义广义特征子空间,然后将广义特征子空间分解为循环子空间的直和,那么我们最直接的想法便是,在所有数域上(包括非代数闭域)定义类似的概念,进行类似的分解. 于是我们首先有如下定义:
\begin{definition}{}{}
    设$V$是一个有限维线性空间,$T\in\mathcal{L}(V)$. 设$T$的特征多项式为
    \[f(\lambda)=p_1(\lambda)^{n_1}p_2(\lambda)^{n_2}\cdots p_k(\lambda)^{n_k},\]
    其中$p_i(\lambda)$是首一不可约多项式,$n_i$是正整数. 对于每一个不可约多项式$p_i(\lambda)$,我们定义\term{广义特征子空间}$G_{p_i}$为
    \[G_{p_i}=\{v\in V\mid\exists m\in\mathbf{N}^+,(p_i(T))^mv=0\},\]
    $G_{p_i}$中的向量则称为\term{广义特征向量}.
\end{definition}

事实上我们很容易发现这就是原先复数域上定义的广义特征子空间在任意数域上的推广形式:当数域为复数域时,$p_i(\lambda)=\lambda-\lambda_i$,其中$\lambda_i$是复数域上的特征值,则$G_{p_i}=\{v\in V\mid\exists m\in\mathbf{N}^+,(T-\lambda_iI)^mv=0\}$,实际上就等同于之前定义的广义特征子空间. 于是,我们也期望将\autoref{thm:广义特征性质} 推广到任意数域上:
\begin{theorem}{}{推广广义特征性质}
    设$V$是有限线性空间,$T\in \mathcal{L}(V)$. $T$的特征多项式有如下分解:
    \[f(\lambda)=p_1(\lambda)^{n_1}p_2(\lambda)^{n_2}\cdots p_k(\lambda)^{n_k},\]
    其中$p_i(\lambda)$是首一不可约多项式,$n_i$是正整数,$G_{p_i}$为广义特征空间,则我们有如下结论:
    \begin{enumerate}[label=(\arabic*)]
        \item 每个$G_{p_i}$在$T$下都是不变的;
        \item $T$对应于不同特征值的广义特征向量线性无关;
        \item $T$不同特征值对应的广义特征子空间的和为直和,且$V=G_{p_1}\oplus G_{p_2}\oplus\cdots\oplus G_{p_k}$;
        \item $V$有一个由$T$的广义特征向量组成的基.
    \end{enumerate}
\end{theorem}
\begin{proof}
    \begin{enumerate}
        \item
        \item
        \item
        \item
    \end{enumerate}
\end{proof}

接下来我们的目标便是将广义特征子空间分解为循环子空间的直和. 事实上,我们无法分解为若当块所需的循环子空间的形式,因为一方面我们不一定在复数域上讨论,另一方面若当标准形有唯一性,而我们需要找到另一种标准形,如果还保持原先的循环子空间还是只能局限于若当标准形. 因此我们有如下定义:
\begin{definition}{}{}
    设$T\in\mathcal{L}(V)$,$v\in V$是一个非零向量. 我们称子空间
    \[W=\{v,Tv,T^2v,\ldots,T^kv,\cdots\}\]
    为\term{由$v$生成的$T$-循环子空间}. 在不引起歧义的情况下,我们也称$W$为\term{$T$-循环子空间}或\term{循环子空间}.
\end{definition}

事实上,$V$一定是$T$的不变子空间,并且当$V$是有限维线性空间时,循环子空间也一定是有限维的. 一个有趣的事实是,如果循环子空间的维数等于$m$,那么它的一组基就是$\{v,Tv,\ldots,T^{m-1}v\}$. 我们来书写这一定理并给出证明:
\begin{theorem}{}{}
    设$V$是有限维线性空间,$T\in\mathcal{L}(V)$,$v\in V$是一个非零向量,$W$是由$v$生成的$T$-循环子空间,则
    \begin{enumerate}
        \item $W$是$T$的不变子空间;
        \item 若$W$的维数为$m$,则$W$的一组基为$\{v,Tv,\ldots,T^{m-1}v\}$.
    \end{enumerate}
\end{theorem}
\begin{proof}
    \begin{enumerate}
        \item
        \item
    \end{enumerate}
\end{proof}

若我们记由$v$生成的$T-\text{循环子空间}$为$C_v$,则我们可以得到限制映射$T|_{C_v}$在循环基$\{v,Tv,\ldots,T^{m-1}v\}$下的矩阵. 为了得到这一矩阵,我们设$Tv=a_0v+a_1Tv+\cdots+a_{m-1}T^{m-1}v$(因为$Tv$一定可以在这组基下被表示),则我们有矩阵表示为
\[A=\begin{pmatrix}
        0      & 0      & \cdots & 0      & -a_0     \\
        1      & 0      & \cdots & 0      & -a_1     \\
        0      & 1      & \cdots & 0      & -a_2     \\
        \vdots & \vdots & \ddots & \vdots & \vdots   \\
        0      & 0      & \cdots & 1      & -a_{m-1}
    \end{pmatrix},\]
事实上我们可以很容易计算出这一矩阵的特征多项式为$|\lambda E-A|=\lambda^m+a_{m-1}\lambda^{m-1}+\cdots+a_1\lambda+a_0$. 更进一步地,根据\autoref{thm:特征多项式等于极小多项式},这一矩阵的极小多项式就等于这一特征多项式. 一件很有趣的事情是,这一矩阵除了主对角线下一行的元素全为1之外,只有最后一列元素非零,且是特征多项式的负系数的排列,因此这一矩阵被称为多项式$\lambda^m+a_{m-1}\lambda^{m-1}+\cdots+a_1\lambda+a_0$的\term{友阵},也称一个\term{弗罗贝尼乌斯块}. 如果一个分块对角矩阵的对角块都是形如上述的弗罗贝尼乌斯块,则这一矩阵被称为\term{弗罗贝尼乌斯标准形},也即\term{有理标准形}.

于是接下来的目标则是将广义特征子空间分解为上述可以生成弗罗贝尼乌斯块的循环子空间的直和. 接下来的几个定理将逐步实现这一目标:
\begin{theorem}{}{}
    若$T$的极小多项式具有如下形式:$g(\lambda)=(p(\lambda))^m$,则$T$有一组基使得其表示矩阵为有理标准形.
\end{theorem}

\begin{corollary}{}{}
    $G_p$有一组由不相交的循环基的并集构成的基.
\end{corollary}

\begin{corollary}{}{第二有理基存在}
    设$V$是$\mathbf{C}$上的有限维线性空间,$T\in\mathcal{L}(V)$,则存在一组基$B$使得$T$在$B$下的矩阵表示为有理标准形.
\end{corollary}
\begin{proof}

\end{proof}
事实上,如果我们按照上一讲的做法将直和分解书写出来,我们发现这样的分解仍然是基于初等因子的分解,我们将这样得到的有理标准形称为\term{第二有理标准形}.

接下来我们需要描述矩阵的第二有理标准形.
\begin{corollary}{}{}
    矩阵的第二有理标准形
\end{corollary}
\begin{proof}

\end{proof}

有理标准形的唯一性定理:
\begin{theorem}{}{}
    \begin{enumerate}
        \item $r_1=\dfrac{1}{d}(\dim V-r(p(T)))$;
        \item $r_i=\dfrac{1}{d}(r(p(T)^{i-1})-r(p(T))^i)$.
    \end{enumerate}
\end{theorem}

\section{不变因子与第一有理标准形}
接下来我们将讨论第二个问题及其回答,从而引入不变因子的概念以及第一有理标准形. 事实上,尽管我们暂未给出第一个问题的解答,但直觉告诉我们,初等因子分解已经是一种比较细的分解方式了——因为它都是一些准素多项式,因此我们希望对一些初等因子进行合并,得到更大的循环子空间. 因此我们需要研究初等因子合并的条件,我们将其叙述为以下定理:
\begin{theorem}{}{初等因子组合}
    设$p_1(\lambda),p_2(\lambda)\in\mathbf{F}[\lambda]$,则循环子空间$C_1\cong\mathbf{F}[\lambda]/(p_1(\lambda))$与$C_2\cong\mathbf{F}[\lambda]/(p_2(\lambda))$可以直和成一个新的循环子空间$C\cong\mathbf{F}[\lambda]/(p(\lambda))$(其中$p(\lambda)\in\mathbf{F}[\lambda]$)当且仅当$p_1(\lambda)$与$p_2(\lambda)$互素,且互素时有$p(\lambda)=p_1(\lambda)p_2(\lambda)$.
\end{theorem}
\begin{proof}
    首先证明必要性. 若$p_1(\lambda)$与$p_2(\lambda)$互素,我们证明$\mathbf{F}[\lambda]/(p_1(\lambda)p_2(\lambda))=\mathbf{F}[\lambda]/(p_1(\lambda))\oplus\mathbf{F}[\lambda]/(p_2(\lambda))$. 回忆证明直和的两种方法,我们选取更简单的,即
    \begin{enumerate}
        \item 交集为$\{0\}$:设$v\in\mathbf{F}[\lambda]/(p_1(\lambda))\cap\mathbf{F}[\lambda]/(p_2(\lambda))$,则基于$p(x)$实际上是$\mathbf{F}[\lambda]/(p(\lambda))$的特征多项式的事实,根据Hamilton-Cayley定理,$p_1(\lambda)v=p_2(\lambda)v=0$. 由于$p_1(\lambda)$与$p_2(\lambda)$互素,因此存在$q_1(\lambda),q_2(\lambda)\in\mathbf{F}[\lambda]$使得$q_1(\lambda)p_1(\lambda)+q_2(\lambda)p_2(\lambda)=1$,因此$v=q_1(\lambda)p_1(\lambda)v+q_2(\lambda)p_2(\lambda)v=0$.
        \item $\dim\mathbf{F}[\lambda]/(p_1(\lambda))+\dim\mathbf{F}[\lambda]/(p_2(\lambda)))=\dim\mathbf{F}[\lambda]/(p_1(\lambda)p_2(\lambda))$:这是显然的,因为由\autoref{ex:多项式域扩张} 可知,$\dim\mathbf{F}[\lambda]/(p(\lambda))=\deg p$,而$\deg p=\deg p_1p_2=\deg p_1+\deg p_2$.
    \end{enumerate}
    接下来证明充分性:我们用反证法,假设$p_1(\lambda)$与$p_2(\lambda)$不互素,且存在$p(\lambda)\in\mathbf{F}[\lambda]$使得$\mathbf{F}[\lambda]/(p(\lambda))=\mathbf{F}[\lambda]/(p_1(\lambda))\oplus\mathbf{F}[\lambda]/(p_2(\lambda))$,我们需要导出矛盾.

    考虑$p_1(\lambda)$和$p_2(\lambda)$的最小公倍式$l(\lambda)|p_1(\lambda)p_2(\lambda)$,且$l(\lambda)\neq p_1(\lambda)p_2(\lambda)$. 然而我们知道,$\forall v\in\mathbf{F}[\lambda]/(p(\lambda))$,由于$v$可以写为$v=v_1+v_2,\enspace v_1\in\mathbf{F}[\lambda]/(p_1(\lambda)),v_2\in\mathbf{F}[\lambda]/(p_2(\lambda))$,因此必有$l(\lambda)v=l(\lambda)v_1+l(\lambda)v_2=0$,因此$l(\lambda)$是$\mathbf{F}[\lambda]/(p(\lambda))$的零化多项式. 而零化多项式是极小多项式的倍式,且循环空间的极小多项式等于特征多项式,故$l(\lambda)$是$p(\lambda)$的倍式. 这表明$\dim\mathbf{F}[\lambda]/(p(\lambda))=\deg p\leqslant \deg l<\deg p_1+\deg p_2$,故根据直和维数相加的等价条件,不可能存在这样的$p(\lambda)$使得$\mathbf{F}[\lambda]/(p(\lambda))=\mathbf{F}[\lambda]/(p_1(\lambda))\oplus\mathbf{F}[\lambda]/(p_2(\lambda))$.
\end{proof}

\begin{corollary}{}{}
    两两互素
\end{corollary}

事实上一个巧合在于这一定理间接回答了第一个问题:
\begin{corollary}{}{}
    最细分解
\end{corollary}
\begin{proof}

\end{proof}

自然地,在证明初等因子分解是最细分解后,我们可以利用\autoref{thm:初等因子组合} 来构造更大的循环子空间. 事实上,根据定理,我们知道这样的构造由非常多种,我们可以仅仅合并两个初等因子,也可以把所有可以合并的初等因子全部合并. 为了得到更标准且优雅的形式,我们做如下合并:首先我们将$T$的特征多项式进行分解得到
\[f(\lambda)=p_1(\lambda)^{n_1}p_2(\lambda)^{n_2}\cdots p_k(\lambda)^{n_k},\]
其中$p_i(\lambda)$是首一不可约多项式,$n_i$是正整数. 于是$T$的初等因子具有形式$p_i^{r_{ij}}$,我们将这些初等因子按照不可约多项式进行分组,即将所有底数$p_i(\lambda)$相同的初等因子写在一行,并且按不可约多项式次数降幂排列:
\begin{gather*}
    p_1^{r_{11}},p_1^{r_{12}},\ldots,p_1^{r_{1s_1}}, \\
    p_2^{r_{21}},p_2^{r_{22}},\ldots,p_2^{r_{2s_2}}, \\
    \cdots \\
    p_k^{r_{k1}},p_k^{r_{k2}},\ldots,p_k^{r_{ks_k}}.
\end{gather*}
其中$r_{ij}$是正整数,$r_{i1}\geqslant r_{i2}\geqslant\cdots\geqslant r_{is_i}$,且有$\prod\limits_{j=1}^{s_i}r_{ij}=n_i$. 接下来我们将每一行的长度对齐,不足的补$1$,于是可以写成
\begin{gather*}
    p_1^{r_{11}},p_1^{r_{12}},\ldots,p_1^{r_{1s}}, \\
    p_2^{r_{21}},p_2^{r_{22}},\ldots,p_2^{r_{2s}}, \\
    \cdots \\
    p_k^{r_{k1}},p_k^{r_{k2}},\ldots,p_k^{r_{ks}}.
\end{gather*}
其中$s=\max\{s_1,s_2,\ldots,s_k\}$. 然后我们从后往前依次计算每一列的元素相乘的结果,即
\begin{align*}
    d_1     & =p_1^{r_{1s}}p_2^{r_{2s}}\cdots p_k^{r_{ks}}, \\
            & \cdots                                        \\
    d_{s-1} & =p_1^{r_{12}}p_2^{r_{22}}\cdots p_k^{r_{k2}}, \\
    d_s     & =p_1^{r_{11}}p_2^{r_{21}}\cdots p_k^{r_{k1}}.
\end{align*}
根据\autoref{thm:初等因子组合},由于$p_i(\lambda)$两两互素,因此我们有
\[\mathbf{F}[x]/(d_{s+1-i})=\mathbf{F}[x]/(p_1^{r_{1i}})\oplus\mathbf{F}[x]/(p_2^{r_{2i}})\oplus\cdots\oplus\mathbf{F}[x]/(p_k^{r_{ki}}),\]
且$\mathbf{F}[x]/(d_i)$可以由一组循环基张成. 因此\autoref{eq:19:初等因子分解} 可以写成经过上述合并的形式:
\begin{equation} \label{eq:20:不变因子分解}
    V=\mathbf{F}[x]/(d_1)\oplus\mathbf{F}[x]/(d_2)\oplus\cdots\oplus\mathbf{F}[x]/(d_s),
\end{equation}
其中$d_i|d_{i+1}(i=1,\ldots,s-1)$. 这样的一组具有``一个整除下一个''性质的多项式被称为$T$的\term{不变因子组},其中的每个多项式$d_i$都是$T$的一个\term{不变因子}.

事实上,由\autoref{thm:初等因子组合} 的互素条件可知,这一分解已经是最``粗粒度''的分解了,因为每个不变因子都是不互素的,无法进一步组合成更大的循环子空间. 接下来我们需要为这一不变因子分解具体实现出一组循环基和对应的标准形.

\begin{theorem}{}{}
    \autoref{thm:初等因子组合} 组合的有理标准形基的构造
\end{theorem}
\begin{proof}

\end{proof}

\begin{theorem}{}{}
    不变因子组也是相似的全系不变量.
\end{theorem}
\begin{proof}
    与初等因子组互相唯一确定,一一对应.
\end{proof}

\begin{corollary}{}{}
    第一有理标准形唯一.
\end{corollary}

\section{关于相似的最后讨论}

% 相抵、相似、相合的全系不变量
最后我们再来总结一个题型. 一些题目可能需要判断矩阵是否相似,实际上我们有如下基本方法:
\begin{enumerate}
    \item 定义法:找到$P$使得$P^{-1}AP=B$即可,这一般是$A,B$没给出具体矩阵的做法,例如上面的性质证明;

    \item 我们也可以先计算两者特征多项式是否相等(即特征值是否一致),若不一致则一定不相似,得到结论,若一致且均为实对称矩阵则相似,否则不一定相似(因为这是相似的必要条件). 对于这种特征值一致的情况,我们进行对角化,情况如下:
          \begin{enumerate}
              \item 若两矩阵均可对角化,则两矩阵相似:因为特征多项式相等则特征值相等,均可对角化那么对角矩阵也完全一致,因此二者与同一个对角矩阵相似,根据相似这一等价关系的传递性可知两矩阵相似;

              \item 若一个矩阵可对角化,另一个矩阵不可对角化,则一定不相似;

              \item 若两个矩阵都不可对角化,不一定相似. 需要两矩阵各个特征值的几何重数(即各个特征子空间维数)都一致才相似,否则不相似. 这是因为只有几何重数一致才有相同的若当标准形.
          \end{enumerate}
\end{enumerate}

\begin{example}{}{}
    设$A,B\in \mathbf{M}_n(\mathbf{F})$,证明:若$A$可逆,则$AB\sim BA$.
\end{example}

\begin{proof}

\end{proof}

\begin{example}{}{}
    设$A=\begin{pmatrix}
            0 & 0 & 1 \\ 0 & 1 & 0 \\ 1 & 0 & 0
        \end{pmatrix},B=\begin{pmatrix}
            -1 & 0 & 0 \\ 0 & 0 & 1 \\ 0 & -1 & 2
        \end{pmatrix}$,判断$A$与$B$是否相似.
\end{example}

\begin{solution}

\end{solution}

\begin{summary}

\end{summary}

\begin{exercise}
    % \exquote[]{}

    \begin{exgroup}
        \item
    \end{exgroup}

    \begin{exgroup}
        \item
    \end{exgroup}

    \begin{exgroup}
        \item
    \end{exgroup}
\end{exercise}
